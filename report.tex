\documentclass[11pt,a4paper]{article}
\usepackage{amsthm}
\usepackage{listings}
\usepackage[usenames,dvipsnames]{color}

\theoremstyle{definition}
\newtheorem*{defn}{Definition}
\definecolor{light-gray}{gray}{0.85}

\title{INF226 Group 5}
\author{Jonny Heggheim \and Samson Gejibo}
\date{\today}

\begin{document}
\maketitle
\newpage
\tableofcontents
\newpage

\lstset{
  language=Java,
  frame=single,
  framexrightmargin=17mm,
  showstringspaces=false,
  backgroundcolor=\color{light-gray}
}

\section{Project Objective}
Aim to introduce static code analysis tools to analyze and understand security
issues relating to development of net-based applications, and be in a position
to utilize programming techniques as defense against different kinds of
security risks.

\begin{defn}
``Static code analysis is the analysis of computer software that is performed
without actually executing programs built from that software.''[4]
\end{defn}
Today, There are different types of analyzer available in the market,
both commercial and open-source. KlockWork, CodeSecure and Fortify can be an
examples of static code analyzers.
In this project, we mainly focused on Fortify static software security analyzer.


\section{What is Fortify?}
Fortify Source Code Analyzer (SCA) is a set of software security analyzers that
 search for violations of security‐specific coding rules and guidelines in a 
 variety of languages. In the analysis information produced by SCA helps us deliver
 more secure software, as well as making security code reviews more efficient, 
 consistent, and complete. This is especially advantageous when large code bases 
 are involved. The modular architecture of SCA allows us to quickly upload new, 
 third party, and customer‐specific security rules. [5]

Since we are familarized with openXdata remote data collection software, from a list 
of  suggested projects, we have chosen OpenXdata source code as an input to 
static software analyzer. 


\section{What is OpenXdata?}
“OpenXdata is a community-developed, open-source, enterprise end-to-end
software solution for handling virtually any type of forms based data
collection and management using mobile devices and web forms
(online or offline).OpenXdata is the first FREE and completely OPEN SOURCE
 end-to-end solution of its kind that can use both high- and low-end mobile 
 devices and mobile phones as well as in a regular web-browser to show and
capture forms based data.”


\section{What is needed to perform analysis?}
\begin{itemize}
 \item Fortify Software with license (fortify.com)
 \item OpenXdata source code (trac.openxdata.org)
 \item Ubuntu 10.10
 \item Eclipse IDE 3.6
\end{itemize}

\section{System Setup}

\subsection{Fortify}

Installation of Fortify Source Code Analysis Suite on Ubuntu 10.10 was straight forward. We used the following steps:
\begin{itemize}
 \item Downloaded the tarball file from: http://software.ii.uib.no/uib/inf226/fortify/ (FrAcademicmLicensence)
 \item We extract the tarball using the following command:
 \begin{verbatim} tar xvzf Fortify-360-2.6.5-Analyzers_and_Apps-Linux-x86.tar.gz\end{verbatim}
 \item Copy the fortify.license file into installation directory.
 \item To update rule packs, under bin directory of Fortify, we run rulepackupdate on the terminal window.
 \item run eclipsepostinstall, under the Fortify/bin folder.
\end{itemize}

Fortify as standalone analyzer and with eclipse have been done successfully without any challenge.
The next step was importing OpenXdata into eclipse IDE.

What is required to build and run OpenXdata?
\begin{itemize}
 \item Java JDK 6
 \item MySQL
 \item Apache Tomcat 6
 \item Maven
 \item (Subversion)
\end{itemize}

\subsection{OpenXdata}
How to build retrieve and build OpenXdata?
Since we the source code is provided by the course, we don’t need to do a checkout from Subversion. So we just need to run `mvn clean install`
How to run OpenXdata?
\begin{enumerate}
 \item Install Apache Tomcat 6 
 \item Install MySQL
 \begin{enumerate}
  \item Create database called openxdata
  \item Grant the default user openxdata/openxdata
 \end{enumerate}
 \item Deploy admin/target/admin.war to Apache Tomcat
 \item OpenXdata will create and populate the database
 \item Open the browser at http://localhost:8080/admin
\end{enumerate}

\section{Critical}

\subsection{Cross-site Scripting: Persistence}
Fortify found 3 security flaws, all of them are false positves.

\begin{lstlisting}[caption=CSVDataExport.java:117]
printWriter.write(line + "\n");
\end{lstlisting}
The produced output is a CSV file, and is not supposed to be executed by the browser.


\begin{lstlisting}[caption=DefaultXformSerializer.java:64]
dos.writeUTF(xml);
\end{lstlisting}
The produced output is a defined binary protocol, the data is read by the mobile client with a parser made by OpenXdata


\begin{lstlisting}[caption=DefaultXformSerializer.java:82]
dos.writeUTF((String)study[1]);
\end{lstlisting}
Same issue as the previous.

\subsection{Cross-Site Scripting: Reflected}
\begin{lstlisting}[caption=StudyExportServlet.java:106]
response.getOutputStream().println(message);
\end{lstlisting}
This is a real issue when the browser executes this as JavaScript.

Message is generated like this
\begin{lstlisting}
request.getParameter("type") + 
    " is not a valid valid type"
\end{lstlisting}
The code is used for error reporting, but it also sets the content type to "text/plain".
This migth reduce the posibility to exploit the security bug.

\subsection{Password Management: Hardcoded Password}
Fortify reports about three issues, all of them are within the tests that are only used
by developers to automated tests. We consider them as false positves since OpenXdata don't
include them for users.

\subsection{Path Manipulation}
\begin{lstlisting}[caption=BirtImagesServlet.java:61]
FileInputStream is = new FileInputStream(
     pathName + request.getParameter("imageName"));
\end{lstlisting}
This allows the attacker to read all files that the servlet container have access to. This issue is very serious.

OTAServlet.java:79
The same as previous, but makes it harder for the attacker.

OpenXDataPropertyPlaceholderConfigurer.java:96 and OpenXDataUtil.java:79 
is false positives since the variable are from the propertyfile that the system administrator provides OpenXdata.

\subsection{Privacy Violation}
DefaultXformSerializer.java:99-102
\begin{lstlisting}
dos.writeByte(users.size());
for(Object[] user : users){
    dos.writeInt((Integer)user[0]);
    dos.writeUTF((String)user[1]);
    dos.writeUTF((String)user[2]);
    dos.writeUTF((String)user[3]);
}
\end{lstlisting}
This is the intention, so we consider this as false positive.

\subsection{Race Condition: Singleton Member Field}
\begin{lstlisting}[caption=ResetPasswordServlet.java:78-82]
mailSender = ctx.getBean("mailSender");
userService = ctx.getBean("userService");
userDetailsService = ctx.getBean("userDetailsService");
messageSource = ctx.getBean("messageSource");
userLocale = new Locale((String)
	request.getSession().getAttribute("locale"));
\end{lstlisting}
The class ResetPasswordServlet is a singleton, so the member fields is shared between users.
The result is that one user could see another user's data.
Since all beans that is provided here is singletons it would not make a different, but for readability it should be fixed.


\begin{lstlisting}[caption=DataExportServlet.java:116 ++]
public void setDataExport(DataExport dataExport) {
    this.dataExport = dataExport;
}
\end{lstlisting}
There are some setters in the Servlets that is used for unit testing. These methods will not be called in production.
We consider them as false positives.

\subsection{SQL Injection}
The SQL is generated like this
\begin{lstlisting}[caption=HibernateFormDownloadDAO.java:113]
"select form_definition_version_id,xform" +
" from form_definition_version fdv" +
" inner join form_definition fd on" +
"   fd.form_definition_id=fdv.form_definition_id" +
" where xform is not null" +
(studyId != null ? " and fd.study_id="+studyId : "") +
    (defaultForms ? " and is_default=1" : "") +
" order by fd.name";
\end{lstlisting}
where studyId is an Integer type, so we dont think that an attacker can inject SQL in this case.


The SQL is generated like this:
\begin{lstlisting}[caption=HibernateFormDownloadDAO.java:149]
"select xform_text from form_definition_version_text "+
"where locale_key='"+ locale +
"' and form_definition_version_id="+formId);
\end{lstlisting}
where locale is a String that is not validated, the attacker needs to be authentication before he can use this for an attack.
The source of the locale variable is: request.getParameter(``locale'')


JdbcRdmsExporterDAO.java + friends
OpenXdata have an exporter that exports submitted forms (XML files) to another database, because of the nature of the task it is hard to use prepared statements.
There are no validations to check that the submitted data contains injected SQL.

\subsection{Header Manipulation}
HTTP headers are control information passed from web clients to web servers on HTTP requests, and from web servers to web clients on HTTP responses. Header manipulation is one of vulnerability in client server interaction. 
Header Manipulation vulnerabilities occur when:
\begin{enumerate}
 \item Data enters a web application through an untrusted source, most frequently an HTTP request.
 \item The data is included in an HTTP response header sent to a web user without being validated.
\end{enumerate}

The analysis tool result shows that header manipulation vulnerability occurs with HTTP response header than request. This can creates different types of attaches such as cache-poisoning, cross-site scripting, cross-user defacement, page hijacking, and cookie manipulation.

DataExportServlet.java:101

In this servlet, the "filename" is an input from the users. On the current openxdata system implementation, it does not validate this input so that there is possible header manuplation attacks such as HTTP response splitting and cross site scripting.

\begin{lstlisting}
response.setHeader("Content-Disposition", "attachment; 
filename=" + filename + ".csv");
\end{lstlisting} 

MultimediaServlet.java:133

This servlet gets two unvalidated inputs request (variable contextType and name) from the users and and perform a responses. There is a potential HTTP response splitting and cross-site scripting attacks.
 
\begin{lstlisting}
response.setContentType(contentType);
response.setHeader
(OpenXDataConstants.HTTP_HEADER_CONTENT_DISPOSITION,
OpenXDataConstants.HTTP_HEADER_CONTENT_DISPOSITION_VALUE + 
name + "\"");
\end{lstlisting} 

ReportServlet.java:68

This servlet does the reporting by taking filename as an input. The current openxdata release do not validate this input. There is a possible Header manupulation attacks.

\begin{lstlisting}[caption= StudyExportServlet.java:95]
response.setHeader("Content-Disposition", "attachment; 
filename=\"" + fileName + "\"");
\end{lstlisting} 
 
 
This servlet takes filename as an input and export study/form as xml data. Here also there is a potential risk of header manuplation attacks.
  
\begin{lstlisting}
response.setHeader("Content-Disposition", "attachment; 
filename=" + filename + ".xml");
\end{lstlisting} 
 
\subsubsection {Head Manuplation attack Mitigation}
 
Input validation takes the first place to mitigate header manuplation attacks. Each input from users must be validated against with malicious code/characters. The openXdata community has considered the potential risk of this attack and now we are working on possible solution to mitigate the attacks. 

\subsection{Insecure Randomness}

Random values are useful in software development and generating true random value has a challenge. “When software generates predictable values in a context requiring unpredictability, it may be possible for an attacker to guess the next value that will be generated, and use this guess to impersonate another user or access sensitive information.”[7]\\\\


In this code , random number has generated using nextLong() and the random number generator implemented by nextLong() cannot withstand a cryptographic attack. 
\begin{lstlisting}[caption=openXdataSecurityUtil.java:175]
public static String getRandomToken() 
{
  Random rng = new Random();
  return encodeString(Long.toString(System.currentTimeMillis()) 
  + Long.toString(rng.nextLong()));
}
\end{lstlisting}

\subsection{Log Forging}
According to Common Weakness Enumeration (CWE) [8], log forging vulnerabilities allows an attacker to forge log entries or inject malicious content into logs and it may occur when data enters an application from an untrusted source and the data is written to an application or system log file.\\\\
Findings from Fortify shows log forging vulnerabilities in the followings code: 


\begin{lstlisting}[caption=DataImportServlet.java:124]
log.info("msisdn " + msisdn + " is user " + msisdnUser.getName());
\end{lstlisting}

\begin{lstlisting}[caption=ImportServlet.java:77]
log.info("Starting import of type: " + importType);
\end{lstlisting}

\begin{lstlisting}[caption=ReExportAllFormData.java:88]
log.info("Found formBinding (for exported table 
name) :"+formBinding);
\end{lstlisting}

\begin{lstlisting}[caption=ReportManager.java:125]
log.info("Creating BIRT engine config with BIRT_HOME = " + 
birtHome);
\end{lstlisting} 
\end{document}