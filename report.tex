\documentclass[11pt,a4paper]{article}
\usepackage{amsthm}
\usepackage{listings}

\theoremstyle{definition}
\newtheorem*{defn}{Definition}

\title{INF226 Group 5}
\author{Jonny Heggheim \and Samson Gejibo}
\date{\today}

\begin{document}
\maketitle
\newpage
\tableofcontents
\newpage

\lstset{
  language=Java,
  frame=single
}

\section{Project Objective}
Aim to introduce static code analysis tools to analyze and understand security
issues relating to development of net-based applications, and be in a position
to utilize programming techniques as defense against different kinds of
security risks.

\begin{defn}
``Static code analysis is the analysis of computer software that is performed
without actually executing programs built from that software.''[4]
\end{defn}
Today, There are different types of analyzer available in the market,
both commercial and open-source. KlockWork, CodeSecure and Fortify can be an
examples of static code analyzers.
In this project, we mainly focused on Fortify static software security analyzer.


\section{What is Fortify?}
Fortify Source Code Analyzer (SCA) is a set of software security analyzers that
 search for violations of security‐specific coding rules and guidelines in a 
 variety of languages. In the analysis information produced by SCA helps us deliver
 more secure software, as well as making security code reviews more efficient, 
 consistent, and complete. This is especially advantageous when large code bases 
 are involved. The modular architecture of SCA allows us to quickly upload new, 
 third party, and customer‐specific security rules. [5]

Since we are familarized with openXdata remote data collection software, from a list 
of  suggested projects, we have chosen OpenXdata source code as an input to 
static software analyzer. 


\section{What is OpenXdata?}
“OpenXdata is a community-developed, open-source, enterprise end-to-end
software solution for handling virtually any type of forms based data
collection and management using mobile devices and web forms
(online or offline).OpenXdata is the first FREE and completely OPEN SOURCE
 end-to-end solution of its kind that can use both high- and low-end mobile 
 devices and mobile phones as well as in a regular web-browser to show and
capture forms based data.”


\section{What is needed to perform analysis?}
\begin{itemize}
 \item Fortify Software with license (fortify.com)
 \item OpenXdata source code (trac.openxdata.org)
 \item Ubuntu 10.10
 \item Eclipse IDE 3.6
\end{itemize}

\section{System Setup}

\subsection{Fortify}

Installation of Fortify Source Code Analysis Suite on Ubuntu 10.10 was straight forward. We used the following steps:
\begin{itemize}
 \item Downloaded the tarball file from: http://software.ii.uib.no/uib/inf226/fortify/ (FrAcademicmLicensence)
 \item We extract the tarball using the following command:
 \begin{verbatim} tar xvzf Fortify-360-2.6.5-Analyzers_and_Apps-Linux-x86.tar.gz\end{verbatim}
 \item Copy the fortify.license file into installation directory.
 \item To update rule packs, under bin directory of Fortify, we run rulepackupdate on the terminal window.
 \item run eclipsepostinstall, under the Fortify/bin folder.
\end{itemize}

Fortify as standalone analyzer and with eclipse have been done successfully without any challenge.
The next step was importing OpenXdata into eclipse IDE.

What is required to build and run OpenXdata?
\begin{itemize}
 \item Java JDK 6
 \item MySQL
 \item Apache Tomcat 6
 \item Maven
 \item (Subversion)
\end{itemize}

\subsection{OpenXData}
How to build retrieve and build OpenXdata?
Since we the source code is provided by the course, we don’t need to do a checkout from Subversion. So we just need to run `mvn clean install`
How to run OpenXdata?
\begin{enumerate}
 \item Install Apache Tomcat 6 
 \item Install MySQL
 \begin{enumerate}
  \item Create database called openxdata
  \item Grant the default user openxdata/openxdata
 \end{enumerate}
 \item Deploy admin/target/admin.war to Apache Tomcat
 \item OpenXdata will create and populate the database
 \item Open the browser at http://localhost:8080/admin
\end{enumerate}

\section{Critical}

\subsection{Cross-site Scripting: Persistence}
Fortify found 3 security flaws, all of them are false positves.

CSVDataExport.java:117
\begin{lstlisting}
printWriter.write(line + "\n");
\end{lstlisting}
The produced output is a CSV file, and is not supposed to be executed by the browser.

DefaultXformSerializer.java:64
dos.writeUTF(xml);
The produced output is a defined binary protocol, the data is read by the mobile client with a parser made by OpenXData

DefaultXformSerializer.java:82
dos.writeUTF((String)study[1]);
Same issue as the previous.

\subsection{Cross-Site Scripting: Reflected}
StudyExportServlet.java:106
response.getOutputStream().println(message);
This is a real issue when the browser executes this as JavaScript.

Message is generated like this
\begin{lstlisting}
request.getParameter("type") + " is not a valid valid type"
\end{lstlisting}
The code is used for error reporting, but it also sets the content type to "text/plain".
This migth reduce the posibility to exploit the security bug.

\subsection{Password Management: Hardcoded Password}
Fortify reports about three issues, all of them are within the tests that are only used
by developers to automated tests. We consider them as false positves since OpenXdata don't
include them for users.

\subsection{Path Manipulation}
from BirtImagesServlet.java:61
\begin{lstlisting}
FileInputStream is = new FileInputStream(pathName + request.getParameter("imageName"));
\end{lstlisting}
This allows the attacker to read all files that the servlet container have access to.

OTAServlet.java:79
The same as previous, but makes it harder for the attacker.

OpenXDataPropertyPlaceholderConfigurer.java:96 and OpenXDataUtil.java:79 
is false positives since the variable are from the propertyfile that the system administrator provides OpenXdata.

\subsection{Privacy Violation}
DefaultXformSerializer.java:99-102
\begin{lstlisting}
dos.writeByte(users.size());
for(Object[] user : users){
    dos.writeInt((Integer)user[0]);
    dos.writeUTF((String)user[1]);
    dos.writeUTF((String)user[2]);
    dos.writeUTF((String)user[3]);
}
\end{lstlisting}
This is the intention, so we consider this as false positive.

\subsection{Race Condition: Singleton Member Field}
ResetPasswordServlet.java:78-82
\begin{lstlisting}
mailSender = (org.springframework.mail.javamail.JavaMailSenderImpl) ctx.getBean("mailSender");
userService = (UserService) ctx.getBean("userService");
userDetailsService = (UserDetailsService) ctx.getBean("userDetailsService");
messageSource = (ResourceBundleMessageSource) ctx.getBean("messageSource");
userLocale = new Locale((String)request.getSession().getAttribute("locale")); //new AcceptHeaderLocaleResolver().resolveLocale(request);
\end{lstlisting}
The class ResetPasswordServlet is a singleton, so the member fields is shared between users.
The result is that one user could see another user's data.
Since all beans that is provided here is singletons it would not make a different, but for readability it should be fixed.


DataExportServlet.java:116 ++
\begin{lstlisting}
public void setDataExport(DataExport dataExport) {
    this.dataExport = dataExport;
}
\end{lstlisting}
There are some setters in the Servlets that is used for unit testing. These methods will not be called in production.
We consider them as false positives.

\subsection{SQL Injection}
HibernateFormDownloadDAO.java:113
The SQL is generated like this
\begin{lstlisting}
"select form_definition_version_id,xform from form_definition_version fdv" +
    " inner join form_definition fd on fd.form_definition_id=fdv.form_definition_id" +
    " where xform is not null" + (studyId != null ? " and fd.study_id="+studyId : "") +
    (defaultForms ? " and is_default=1" : "") + " order by fd.name";
\end{lstlisting}
where studyId is an Integer type, so we dont think that an attacker can inject SQL in this case.


HibernateFormDownloadDAO.java:149
The SQL is generated like this:
\begin{lstlisting}
"select xform_text from form_definition_version_text where locale_key='"+ locale + "' and form_definition_version_id="+formId);
\end{lstlisting}
where locale is a String that is not validated, the attacker needs to be authentication before he can use this for an attack.
The source of the locale variable is: request.getParameter(``locale'')


JdbcRdmsExporterDAO.java + friends
OpenXdata have an exporter that exports submitted forms (XML files) to another database, because of the nature of the task it is hard to use prepared statements.
There are no validations to check that the submitted data contains injected SQL.


\end{document}